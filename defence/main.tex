
\documentclass[]{beamer}
%\usepackage{pgfpages}
%\pgfpagesuselayout{2 on 1}[a4paper,border shrink=5mm]
%\setbeameroption{show notes}

\usepackage{ctex}
\usepackage{siunitx}
\usepackage{graphicx}

\usepackage[
  backend=biber,
  style=numeric,
  citestyle=ieee,
  bibencoding=utf8,
]{biblatex}
\addbibresource{main.bib}

\setbeamertemplate{caption}{\raggedright\insertcaption\par}

\usetheme{Montpellier}
\usecolortheme{default}

\title[硕士学位论文答辩]{大跨越输电塔结构在龙卷风作用下的响应分析}
\author{140926~王勇 \\[1.5 \baselineskip]  \footnotesize{导师: 吕令毅 ~ 教授}}
\institute{东南大学土木工程学院}
\date{\today}

\begin{document}

\begin{frame}
  \titlepage
\end{frame}

\section*{目录}
\begin{frame}
  \tableofcontents
\end{frame}

\section{引言}
\subsection{课题研究背景与意义}

\begin{frame}
  \frametitle{课题研究背景与意义}
  \begin{itemize}
  \item
    大跨越输电塔结构是重要的生命线电力工程设施 ,具有数量大、分布广等特点,容易遭受龙卷风的袭击。
  \item
    全球范围内约 $80\%$ 的输电塔倒塌破坏是由于极端天气(如龙卷风、雷暴等)的影响。
  \item
    输电塔结构破坏将导致供电系统的瘫痪甚至引发火灾等严重后果,造成重大经济损失。
    \bigskip
  \item
    鉴于近年龙卷风发生频率和强度似有增大趋势,而以往国内针对输电塔受龙卷风袭击的研究较少,故为保障龙卷风多发地区电网运行安全,进行输电线路的龙卷风荷载及抗风研究具有重要的理论和实用价值。
  
  \end{itemize}
 
\end{frame}
\subsection{课题研究思路与内容}

\section{龙卷风风场及其数值模拟}
\subsection{缩尺龙卷风的CFD模拟}
\subsection{缩尺龙卷风数值模拟结果及正确性验证}
\subsection{足尺龙卷风的CFD模拟}

\section{输电塔结构在龙卷风作用下的静态响应分析}
\subsection{背景工程介绍}
\subsection{输电塔龙卷风荷载计算的FSI方法}
\subsection{输电塔龙卷风荷载计算的规范方法}
\subsection{FSI方法与规范方法的对比}

\section{考虑龙卷风平移效应的动态响应分析}
\subsection{动态龙卷风模型}
\subsection{动态龙卷风风速和荷载时程}
\subsection{输电塔结构动力时程分析}

\section{课题总结与展望}

\section{致谢}

\end{document}