\chapter{Wind-Driven Rain}

\section{Overview}
\textbf{Euler-Euler frame in WDR simulation}\cite{Huang2011WDR}: 
rain is regarded as a continuum similar to wind rather than as a single raindrop as it is treated in the existing WDR simulation methods. 
A similar group of conservation equations with wind, including the mass conservation and momentum conservation equations, 
were established for rain based on the concept of phase and phasic volume fraction in the same Euler coordinates as wind. 
Then, the equations for rain can be solved in the same way as for wind.

\textbf{Continuum hypothesis for raindrops}\cite{Huang2010WDR}:
Continuum hypothesis is valid mathematically in consideration of the relatively small mean inter-particle distance with an order of \SI{0.1}{m}
in most cases, as compared with the macroscopic size of a concerned WDR field (usually in the order of \SI{10}{m}).


\section{Governing equations for WDR with Eulerian multiphase model}
\textbf{Phase in multiphase flow regime}\cite{Huang2010WDR}:
In a multiphase flow regime, \textit{phase} does not only refer to the physical phase of material, such as gas, liquid and solid,
but also it has been defined in a broder sense as an identifiable class of material that has a particular inertial response to
and interaction with the flow and the potential field in which it is immersed.
Specifically in the WDR description, in addition to wind (air) phase (gas phase in physics), 
different-sized raindrops can be regarded as different phases because each collection of raindrops with the same size will have a similar dynamical response to the wind field.

\textbf{Phasic volume fraction $a_q$}:
represents the space occupied by each phase, which is regarded as a continuous function in space and time.

Rain is divided into $\mathnormal{N}$ phases according to the diameter sizes of raindrops, 
with each phase representing the raindrops' diameter being in the range of $\left[D_k-(dD/2), D_k+(dD/2)\right]$,
where $dD$ is the differential diameter range and $k=1,2,\cdot,N.$
The volume fraction of the $k$th phase, denoted by $a_k$, 
represents the volume fraction of raindrops belonging to the same diameter range $\left[D_k-(dD/2), D_k+(dD/2)\right]$.

\textbf{Mass conservation equation and Momentum conservation equation for the phase $k$}:
\begin{equation}
\frac{\partial \rho_l a_k}{\partial t}+\frac{\partial \left(\rho_l a_k u_{kj}\right)}{\partial x_j} = 0
\end{equation}
\begin{equation}\label{eqn:monmentum}
\frac{\partial \rho_l a_k u_{ki}}{\partial t} + \frac{\partial \left(\rho_l a_k u_{ki} u_{kj}\right)}{\partial x_j} =
\rho_l a_k g_i + \rho_l a_k \frac{18\mu C_D Re_p}{24 \rho_l D_k^2}\left(u_i-u_{ki}\right)
\end{equation}
\begin{description}
\item[$\rho_l$: ] the physical density of rain
\item[$u_{ki}$: ] the absolute velocity component of the $k$th phase of rain
\item[The first term in the right hand side of Eq. \ref{eqn:monmentum}: ] the source of gravitational force, where $g_i$ represents the gravity in the $i$th direction
\item[The second term in the right hand side of Eq. \ref{eqn:monmentum}: ] denotes the source of drap force between the rain and the wind phases, in which $Re_p$ is the relative Reynolds number (referring to the wind around the raindrop); $C_D$ is the raindrop drag coefficient which is usually determined by fitting measurements, using a polynomial formula. $\mu$ and $\mu_i$ are the viscosity coefficient and velocity component of wind, respectively.
\end{description}

\textbf{For \textit{wind flow}, RANS equations, in combination with the realizable $k-\epsilon$ turbulence model:}\par 
