\documentclass{ctexart}

\usepackage{geometry}
\geometry{
  a4paper,
  total={210mm,297mm},
  left=20mm,
  right=20mm,
  top=20mm,
  bottom=20mm,
}

\usepackage{hyperref}
\hypersetup{
  bookmarksnumbered=true,
  colorlinks=true,
  allcolors=blue,
}

\usepackage{siunitx}


\title{低矮房屋的体型对其所受龙卷风荷载影响的的研究}
\author{Jeremy Case}
\date{}

\begin{document}

\maketitle

\begin{abstract}
尽管龙卷风的破坏力巨大,但仅存在有限的尝试试图定量分析龙卷风引起的荷载。
本文的目的在于分析建筑的不同体型对其在模拟龙卷风(模拟龙卷风的涡流比根据实测龙卷风取值)作用下所受荷载和风压的影响。
测量得到的建筑所受荷载和风压被用于评估抗龙卷风设计的可行性。
龙卷风实验装置核心直径\SI{56}{m},能产生较大的涡流比(\num{2.6})以代表EF3等级龙卷风,建筑的缩尺模型放置于实验装置中,以测量其所受风压。
研究表明,最大风压与檐口高度、屋顶高度、纵横比、平面面积、及其它建筑几何形状的变化(如附加的车库、挑檐和拱腹)有关。
根据测量所得风压计算屋盖与墙、墙板与椽子的连接所需强度,并与其实际强度进行比较,以评估结构失效的概率。
结果表明,住宅中两处关键的连接(主要是屋盖与墙的连接)若设计得当,则能保证足够的安全储备以抵抗不超过EF3等级的龙卷风的袭击。
\end{abstract}

\section{引言}
许多观点质疑住宅抗龙卷风设计的可能性,遑论其可行性。
这样的观点大多来自于对结构在强风作用下的破坏程度的评估,并因龙卷风惊人的破坏力而产生偏见,但这些观点并未依据足尺或缩尺建筑的龙卷风试验所测压力与结构构件与连接具有的实际抗力的比较。
现代工程学基于理论和经验总结的准则,但对于抗龙卷风设计,几乎没有数据能用于形成实用的工程准则。
过去,在龙卷风袭击时,屋盖所受风压及结构所受荷载的近似数据仅被用于法院的调查和工程的判断之中。
这种限制来源于三个困难:缺乏能够测量龙卷风引起的风压和荷载的仪器;缺乏实测数据与实验数据进行对比研究;以及缺乏对抗龙卷风设计的兴趣,因为人们假定这样做是不经济的。

为了克服上文第一个困难,爱荷华州立大学(Iowa State University, ISU)建成了一个龙卷风模拟装置。
ISU龙卷风装置能产生相对于建筑模型运动的龙卷风,也能调整多个参数以产生不同种类的龙卷风。
随着近年来几次龙卷风的实测数据的陆续发表,第二个困难也能迎刃而解了。
随后的工作比较了ISU实验室模拟的龙卷风与实测龙卷风及计算流体力学的龙卷风模型的特征。
Thampi等人的工作表明:当把低矮建筑所受ISU模拟龙卷风的风压输入到建筑的有限元模型中,相应的足尺建筑所受的破坏能够重现。
这些进步使得结构所受龙卷风荷载及风压的确定成为可能,并能据此探究经济可行的抗龙卷风设计。

本文研究了低矮建筑受到龙卷风引起的风压和荷载与建筑的几何形状及朝向的关系。
还研究了轻型木框架结构屋盖处两个最为脆弱的连接在实验室所测龙卷风风压作用下的抗掀能力,以探究抗龙卷风设计的可行性。




\end{document}
