\begin{abstract}{龙卷风数值模拟;大跨越输电塔结构;静力弹塑性分析;参数化分析;动力时程分析}
  % 摘要写法
  %% 第一段:研究对象介绍;研究意义;目前研究不足之处;本文主要研究问题
  %% 第二段-第四段:
  %%% 本章主要研究问题;
  %%% 采用的研究方法;
  %%% 主要结果。
  龙卷风是一种伴随着高速旋转的漏斗状云柱的强风涡旋,发生的概率较低,但破坏力巨大。
  大跨越输电塔结构是重要的生命线电力工程设施,具有数量大、分布广等特点,容易造成龙卷风的袭击,其破坏将导致供电系统的瘫痪甚至引发火灾等严重后果,造成重大经济损失。
  美国输电塔设计规范已考虑龙卷风等极端天气荷载,我国相关设计规范尚未涉及。 为保障龙卷风多发地区电网运行安全,进行输电线路的龙卷风荷载及抗风研究具有重要的理论和实用价值。
  本文利用计算流体动力学(CFD)模拟技术和规范方法,研究龙卷风数值风场的生成和其作用于大跨越输电塔结构的静态荷载和考虑龙卷风移动效应的动态荷载,分析典型工况下输电塔结构的响应。
  
  首先,研究利用CFD模拟技术建立龙卷风数值风场的方法。
  根据龙卷风的试验模拟文献,并参考前人的数值模拟方法,建立了相应于缩尺龙卷风发生装置的CFD数值模型。
  模拟结果表明数值风场的切向速度的分布与Rankine涡模型及试验测量的风速分布吻合较好,验证了缩尺龙卷风数值风场的可靠性。
  并将缩尺龙卷风数值风场与1998年发生在Spencer地区的龙卷风实测风场进行对比,通过长度相似比和速度相似比的概念将缩尺风场模型改造成足尺模型。
  发现当长度相似比取为$4000$、速度相似比取为$60$时,龙卷风足尺数值风场与实测风场的切向速度分布吻合较好。
  
  其次,利用单向流固耦合(FSI)方法和规范方法计算龙卷风作用于输电塔结构的荷载,并分析比较结构响应。
  FSI方法直接在足尺龙卷风风场中建立输电塔结构的刚性模型,通过CFD方法计算得到结构表面受到的龙卷风风压,并将其转化为输电塔梁单元模型的节点集中力;
  规范方法利用输电塔设计规范中风荷载计算公式将足尺龙卷风数值风场转化为风荷载。
  然后在龙卷风典型袭击角度(\SI{0}{\degree}、\SI{45}{\degree}和\SI{90}{\degree})工况下,以两种方法计算输电塔结构受到的龙卷风荷载,进行静力弹塑性分析并分析结构响应。
  结果表明,采用FSI 方法计算得到的输电塔结构响应整体上小于规范方法,且最大轴向力误差不超过$20\%$,位移响应不超过$30\%$;
  两种方法计算的结构响应随龙卷风袭击角度的变化趋势相似,可实现两种龙卷风荷载计算方法的相互验证。
  
  最后,进行了输电塔结构在考虑龙卷风平移效应时的动态响应分析。
  选取龙卷风移动路径平行和垂直于输电线的两种典型工况,在任意时间步利用规范方法计算并施加龙卷风动态荷载,进行动力时程分析,并分析龙卷风动态荷载和结构典型响应的时程。
  结果表明,平行工况下龙卷风荷载总和时程在输电线方向的分量出现持续时间约为\SI{10}{s}的波形(荷载先逐渐增大,后逐渐衰减),且荷载峰值较大,呈现出一定的冲击效应。
  垂直工况下龙卷风荷载总和时程在垂直输电线方向的分量出现类似的特点。
  且在两种工况下,输电塔塔顶节点位移时程与龙卷风荷载总和时程变化趋势相似。

\end{abstract}

\begin{englishabstract}{}

\end{englishabstract}

%%% Local Variables:
%%% mode: latex
%%% TeX-PDF-mode: t
%%% TeX-engine: xetex
%%% End:
