
\chapter{结论与展望}

\section{全文总结}

本文利用CFD数值模拟技术,分析了龙卷风作用下输电塔结构的准静态和动态响应。
选取\SI{500}{kV}南京三江口长江大跨越工程中跨越塔为研究对象,分别以单向流固耦合方法和规范方法计算结构受到的龙卷风荷载,分析比较龙卷风袭击角为\SI{0}{\degree}、\SI{45}{\degree}时的准静态响应;
并考虑龙卷风的平移运动,进行输电塔结构在龙卷风动态荷载作用下的时程分析。
本文的研究成果可为输电塔结构的抗龙卷风设计提供一定的参考。
本文的主要研究内容和结论可概括为以下几点:

1)简要介绍了龙卷风这种破坏力巨大的极端气象灾害,包括其风场特性及造成的经济损失。
综述了当前龙卷风研究方法的现状,包括现场实测、试验模拟和数值分析。
根据目前国内输电塔结构抗龙卷风研究的现状,本文说明了研究课题的意义和主要的研究思路。

2)依据文献中龙卷风试验模拟为基础建立了CFD缩尺龙卷风风场,发现缩尺风场的切向风速的分布与Rankine涡模型及试验测得的风速分布吻合较好,呈现的风速分布特征为:
切向速度在涡旋中心处较小,随着离涡旋中心的距离增大而增大,在核心半径处达到最大,而后随着远离涡旋中心的距离增大而逐渐减小;
且沿径向分布时,在核心半径内,切向速度变化较快,而远离核心半径时,变化逐渐缓和。
然后将缩尺龙卷风数值风场与1998年发生在Spencer地区的实测龙卷风风场进行对比,通过引入长度相似比和速度相似比的概念将缩尺缩尺风场模型改造成足尺龙卷风风场模型。
发现速度相似比取为$60$、根据文献取长度相似比为$4000$时,龙卷风足尺数值风场与Spencer龙卷风实测风场吻合较好。

3)探讨了龙卷风袭击角分别为$\SI{0}{\degree}$、$\SI{45}{\degree}$和$\SI{90}{\degree}$工况时输电塔结构的静态响应。
分别利用单向流固耦合方法和规范方法计算龙卷风荷载,分析比较结构的最大轴向力和位移响应。
发现采用单向流固耦合和规范方法计算得到的输电塔结构响应整体上吻合较好,按两种方法计算的结构轴向力和位移响应随龙卷风袭击角的变化趋势相似,且最大轴向力误差不超过$20\%$,最大位移响应误差不超过$30\%$,可以互相验证两种龙卷风荷载计算方法。
\SI{45}{\degree}工况下结构在$X$和$Y$方向均发生变形,会出现整体的扭转变形。
单向流固耦合方法和规范方法计算得到的输电塔结构在龙卷风作用下的代表性构件的轴力,发现梁柱等关键构件轴力误差较小,支撑等次要构件轴向力响应误差不超过$50\%$。

4)考虑龙卷风的平移运动,  选取龙卷风移动路径平行和垂直于输电线的两种典型工况,在任意时间步利用规范方法计算并施加龙卷风动态荷载,进行动力时程分析,并分析龙卷风动态荷载和结构典型响应的时程。
结果表明,平行工况下龙卷风荷载总和时程在输电线方向的分量出现持续时间约为\SI{10}{s}的波形(荷载先逐渐增大,后逐渐衰减),且荷载峰值较大,呈现出一定的冲击效应。
垂直工况下龙卷风荷载总和时程在垂直输电线方向的分量出现类似的特点。
且在两种工况下,输电塔塔顶节点位移时程与龙卷风荷载总和时程变化趋势相似。


\section{研究展望}

龙卷风是一种小尺度气旋,破坏力极强,近年来成为风工程领域里研究的热点。
龙卷风的数值模拟方法克服其他方法,如现场实测和试验模拟等的一些不足之处,具有方便、快速、经济和高效率等优势。
本文利用CFD数值方法模拟了输电塔结构受到的龙卷风荷载,并改变龙卷风核心相对于输电塔结构的位置,进行龙卷风作用下的准静态分析和动力时程分析。
本文得到一些建设性的结论,但仍然存在许多不完善的地方。
今后可考虑从以下几个方面进行完善。

1)实测的龙卷风存在多涡形态,且受地面粗糙度,研究者认为实测的龙卷风涡流比约为$2.0$,本文模拟的风场涡流比为$0.28$,故本文模拟的数值龙卷风与实际情况存在差距。
今后可考虑通过引入更为精细的数值模拟方法(如大涡模拟等),以模拟出更为符合实际的龙卷风风场。

2)本文忽略了输电线受到的龙卷风作用,这主要是考虑到龙卷风尺度相比输电线长度较小,且输电线受到的风荷载作用机理较为复杂。
本文还忽略了输电线与塔的耦合作用,这种耦合作用在动力时程分析中会产生较大的阻尼,但考虑这一作用会引起结构有限元分析的较强非线性。
今后的研究可考虑输电线受到的龙卷风荷载和输电线与塔的耦合作用。

3)本文中龙卷风作用下输电塔结构的准静态分析考虑了结构材料非线性和几何非线性,但考虑龙卷风移动效应的动力时程分析中假定结构处于弹性状态。
这主要是由于动力时程分析计算量较大,笔者不具有足够的计算资源实施动力弹塑性分析。
今后的研究可考虑动力时程分析中结构的材料非线性和几何非线性,进而进行倒塌性分析。

4)真实的龙卷风风场中风致飞射物对结构的打击作用不容忽视,也是造成结构破坏的一个重要因素。
本文在此未考虑风致飞射物的影响。
今后的研究中可根据龙卷风强度,对不同形状、不同质量的飞射物的运动轨迹进行模拟,以确定风致飞射物对建筑结构的影响。


%%% Local Variables:
%%% mode: latex
%%% TeX-PDF-mode: t
%%% TeX-engine: xetex
%%% TeX-master: "../main"
%%% End:
