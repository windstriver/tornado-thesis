
转眼间已到了毕业季,也是我在南京生活的第七年了。
七年的青春先后在东南大学九龙湖校区和四牌楼校区度过,所学所思,伴我一生。

感谢导师吕令毅教授,引导我寻找课题、比较研究方法,谆谆教诲「做人做事做学问」的理念。
吕老师治学严谨求实,生活简单健康,为人质朴淡泊名利,亦追求卓越,不断进取。
若无吕老师的悉心启迪与引领,难有我论文顺利完成。
在此,万分感谢恩师。

感谢江苏省电力设计院张晋绪师兄提供南京三江口大跨越输电塔工程的设计参数,
并指导了输电塔结构的相关知识。

良好的课题组氛围令我的论文工作充满了思维碰撞的乐趣。
感谢已走上工作岗位的师兄(姐)刘道勇、王宁、宋拓、汤卓、唐飞燕、彭英海、张弛的辛勤指导,帮助良多。
师兄徐康,师弟张良辰、胡婉亭、吴熙、刘远之,同门许逸文、王美珍朝夕相处,探讨课题,获益匪浅。
文昌十二舍601,陈凯、刘震、朱冬平自本科便相识,相伴近六年,时而鼓励,催我奋进,时而倾听,解我烦忧,已成一生挚友。

论文准备和撰写期间使用了许多优秀的开源软件,感谢其开发者和社区维护者,他们的工作令论文的书写多了优雅和享受,包括Emacs编辑器,{\LaTeX} 排版系统和Python编程语言。

最后感谢我平凡伟大的父母和亲爱的弟弟,他们的支持与奉献,给了我无穷的勇气和毅力完成学业。

%%% Local Variables:
%%% mode: latex
%%% TeX-PDF-mode: t
%%% TeX-engine: xetex
%%% TeX-master: "../main"
%%% End:
