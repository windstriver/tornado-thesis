\chapter{风压转化为梁节点集中力的Python程序}\label{apen:load}

\section{main.py}
\begin{minted}{python}
import numpy as np

import reader as rd
import crossSele as cs
import pNormIJ as pIJ

loadOut = 'load.csv'
epsilon = 0.01

# Initiate Matrix elePres to hold pressure on
# elemnt (eleNum, presX, presY, presZ)
elePres = np.zeros((rd.elemCount, 5))
elePres[:, 0] = range(1,rd.elemCount+1)

print("Start to map the CFD Pressure Points to Elements...")

for p in range(rd.presCount):
    # p iterate all the cfd pressure points
    XP = rd.cfdPresM[p, 0:3]    # Coordinates of point p+1
    pres = rd.cfdPresM[p, 3]    # pressure of point p+1
    
    # decide cfd pressure point (p+1) belongs to which element
    for e in range(rd.elemCount):
        # e interate all elements, e refers element (e+1)
        (dist, normVec, flag) = pIJ.pNormIJ(XP, cs.XIE(e+1), cs.XJE(e+1),\
            distEpsilon=cs.RoE(e+1)*(1+epsilon))   
        # compute the distance from point (p+1) to element (e+1)
        if flag:
            # cfd pressure point (p+1) belongs to element (e+1)
            elePres[e, 1:4] = elePres[e, 1:4] + pres*normVec
            elePres[e, 4] = elePres[e, 4] + 1
            print("CFD Pressure Point ({X:9.3f},{Y:9.3f},{Z:9.3f}) \
               belongs to Element {eleNum:6d}"\
               .format(X=XP[0], Y=XP[1], Z=XP[2], eleNum=e+1))
            break

print("CFD Pressure Points mapping ratio:{:6.3%}."\
  .format(sum(elePres[:,4])/rd.presCount))

print("Starting convert the CFD Pressure to Elemeng nodal force...")
# Transfer the element pressures to element node force
# Init nodeForce matrix to hold force on nodes (nodeNum, Fx, Fy, Fz)
nodeForce = np.zeros((rd.nodeCount, 4))
nodeForce[:, 0] = range(1,rd.nodeCount+1)

for e in range(rd.elemCount):
    # (e+1) interate all elements
    if elePres[e, 4] == 0:
        presMean = np.zeros((3,))
    else:
        presMean = elePres[e, 1:4] / elePres[e, 4]

    forceTotal = presMean * cs.AoE(e+1)
    nodeI = int(rd.eleM[e, 1])
    nodeJ = int(rd.eleM[e, 2])
    nodeForce[nodeI-1, 1:4] = nodeForce[nodeI-1, 1:4] + forceTotal/2
    nodeForce[nodeJ-1, 1:4] = nodeForce[nodeJ-1, 1:4] + forceTotal/2
    print("Element {eleNum:6d} nodal force \
                   F=({Fx:9.3f} N, {Fy:9.3f} N, {Fz:9.3f} N)"\
      .format(eleNum=e+1, Fx=forceTotal[0]/2, \
              Fy=forceTotal[1]/2, Fz=forceTotal[2]/2))


print("Sum FX is {:6.3f} N.".format(sum(nodeForce[:, 1])))
print("Sum FY is {:6.3f} N.".format(sum(nodeForce[:, 2])))
print("Sum FZ is {:6.3f} N.".format(sum(nodeForce[:, 3])))


with open(loadOut, 'w') as f:
    for n in range(rd.nodeCount):
        f.write('{NODE:6d}, {Fx:9.3f}, {Fy:9.3f}, {Fz:9.3f}\n'\
        .format(NODE=n+1, Fx=nodeForce[n, 1], Fy=nodeForce[n, 2],\
                Fz=nodeForce[n, 3]))

f.close()

\end{minted}

\section{reader.py}
\begin{minted}{python}
import numpy as np

inputDir = "Input/"
caseDir = "theta-0/"
# Read CFD Pressure Data (x[m], y[m], z[m], pres[Pa])
cfdPresFile = inputDir + caseDir + "cfd_pressure.csv"
cfdPresM = np.genfromtxt(cfdPresFile, delimiter=',', skip_header=17)
cfdPresM[:,3] = 1.0*cfdPresM[:,3]    # into the surface pressure +

# Read CFD velocity Data
# (nodeNum, x[m], y[m], z[m], pres[Pa], u[m/s], v[m/s], w[m/s])
cfdVeloFile = inputDir + caseDir + "cfd_velocity2.out"
cfdVeloM = np.genfromtxt(cfdVeloFile, delimiter=',')

# Remove the zero lines
cfdV = cfdVeloM[~np.all(cfdVeloM == 0.0, axis = 1)]
cfdV = cfdV[:, [0,5,6,7]]

# Read Node List File (nodeNum, nodeX[m], nodeY[m], nodeZ[m])
nodeFile = inputDir+"node_list.txt"
nodeM = np.genfromtxt(nodeFile, delimiter=',')

# Read Element List File (eleNum, eleI, eleJ, eleSecn)
eleFile = inputDir+"ele_list.txt"
eleM = np.genfromtxt(eleFile, delimiter=',')

# Read Section List File (secNum, Ro[m])
secFile = inputDir+"sec_list.txt"
secM = np.genfromtxt(secFile, delimiter=',')

presCount = np.size(cfdPresM, 0)    # total num. of cfd pressure points
nodeCount = np.size(nodeM, 0)    # total num. of nodes
elemCount = np.size(eleM, 0)    # total num. of elements
secCount = np.size(secM, 0)    # total num. of sections
veloCount = np.size(cfdV, 0)

\end{minted}

\section{pNormIJ.py}
\begin{minted}{python}
import numpy as np

def pNormIJ(CoordP=(0.0, 0.0, 0.0), CoordI=(1.0, 0.0, 0.0), \
                    CoordJ=(0.0, 1.0, 0.0), distEpsilon=1.0):
    """ Return the distance of point P to line IJ and the norm vector PP'
        where P' is the projection of P to IJ.

    Keyword arguments:
    CoordP -- the coordinates of point P, default (0.0, 0.0, 0.0)
    CoordI -- the coordinates of point I, default (0.0, 1.0, 0.0)
    CoordJ -- the coordinates of point J, default (0.0, 0.0, 1.0)
    distEpsilon -- distance criterion

    Return:
    dist -- distance from point P to line IJ
    normVec -- norm vector of PP'
    """
    CoordP = np.array(CoordP)
    CoordI = np.array(CoordI)
    CoordJ = np.array(CoordJ)
    CoordPPrime = np.zeros((3,))
    flag = False

    normIJ = np.linalg.norm(CoordI - CoordJ)
    CoordPPrime = CoordI + np.dot(CoordI-CoordP,\
        CoordI-CoordJ)/(normIJ**2)*(CoordJ-CoordI)
    dist = np.linalg.norm(CoordP-CoordPPrime)
    normVec = (CoordPPrime-CoordP)/dist

    if (np.dot(CoordPPrime-CoordI, CoordPPrime-CoordJ) < 0 and\
        dist < distEpsilon):
        flag = True

    return (dist, normVec, flag)
\end{minted}

\section{crossSele.py}  
\begin{minted}{python}
import numpy as np

import reader as rd

def XIE(eleNum=1):
    """ Get the coordinates of I node of Element eleNum.

    Keyword argument:
    eleNum -- element number
    """
    eINum = int(rd.eleM[eleNum-1][1])
    return rd.nodeM[eINum-1][1:]


def XJE(eleNum=1):
    """ Get the coordinates of J node of Element eleNum.

    Keyword argument:
    eleNum -- element number
    """
    eJNum = int(rd.eleM[eleNum-1][2])
    return rd.nodeM[eJNum-1][1:]


def RoE(eleNum=1):
    """ Get the Ro of section of Element eleNum.

    Keyword argument:
    eleNum -- element number
    """
    secNum = int(rd.eleM[eleNum-1][3])
    return rd.secM[secNum-1][1]

def AoE(eleNum=1):
    """ Get the surface area of Element eleNum.

    Keyword argument:
    eleNum -- element number    
    """
    length = np.linalg.norm(XIE(eleNum)-XJE(eleNum))
    return 2*np.pi*RoE(eleNum)*length

\end{minted}

%%% Local Variables:
%%% mode: latex
%%% TeX-master: "../main"
%%% End:
