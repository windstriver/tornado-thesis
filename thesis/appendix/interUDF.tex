\chapter{CFD风场任意位置处风速的UDF插值程序}\label{apen:inter}
\begin{minted}{c}
#include "udf.h"

#define MAXPOINTS 5000
#define NSCALARS (1+ND_ND)

real coords[MAXPOINTS][ND_ND+1];
real values[MAXPOINTS][NSCALARS+ND_ND+1];

int total_count;
int total_points_found = 0;

FILE *input;
FILE *output;

int is_point_in_cell(cell_t c, Thread *t, real *x);

DEFINE_ON_DEMAND(interpolate)
{
    Domain *d = Get_Domain(1);
    Thread *t;
    cell_t c;

    real NV_VEC(x), NV_VEC(dx), NV_VEC(grady), NV_VEC(centroid);
    real y0, y;

    int point, i, n;
    int reader_flag;

    /* Read the Points data to array coords */
    input = fopen("node_list_0.txt", "r");

    for(i=0; i<MAXPOINTS; i++)
    {
        reader_flag = fscanf(input, "%f, %f, %f, %f\n",\
            &coords[i][0], &coords[i][1], &coords[i][2], &coords[i][3]);

        if (reader_flag!=4)
            break;

        total_count = i;
    }
    fclose(input);


    /* Initialize values */
    for(i=0; i<MAXPOINTS; i++)
    {
        for(n=0; n<NSCALARS+ND_ND+1; n++)
        {
            values[i][n] = 0.0;
        }
    }
    
    /* Interpolation starts ... */
    for(point=0; point<total_count; point++)
    {
        /* Get coords of point to vec x */
        for(i=0; i<ND_ND; i++)
            x[i] = coords[point][i+1];

        /* Find the cell of point */
        thread_loop_c(t, d)
        {
            begin_c_loop(c, t)
            {
                if (is_point_in_cell(c, t, x))
                {
                    C_CENTROID(centroid, c, t);
                    total_points_found++;
                    Message("The cell including point (%f,%f,%f) has\
                        centroid of (%f,%f,%f).\n",\
                        x[0],x[1],x[2],centroid[0],centroid[1],centroid[2]);

                    /* Store the coords */
                    values[point][0] = coords[point][0];
                    for(i=0; i<ND_ND; i++)
                    {
                        values[point][i+1] = x[i];
                    }

                    y0 = 0.0;
                    NV_VV(dx, =, x, -, centroid);
            
                    for(n=0; n<NSCALARS; n++)
                    {
                        switch (n)
                        {
                            case 0:
                                NV_V(grady, =, C_P_G(c,t));
                                y0 = C_P(c,t);
                                break;
                            case 1:
                                NV_V(grady, =, C_U_G(c,t));
                                y0 = C_U(c,t);
                                break;
                            case 2:
                                NV_V(grady, =, C_V_G(c,t));
                                y0 = C_V(c,t);
                                break;
                            case 3:
                                NV_V(grady, =, C_W_G(c,t));
                                y0 = C_W(c,t);
                                break;
                            default:
                                break;
                        }

                        y = y0 + NV_DOT(grady, dx);
                        values[point][ND_ND+n+1] = y;
                    }
                    break;
                }
            }
            end_c_loop(c, t)
        }
    }

    output = fopen("output.out","a");
    if (!output)
    {
        Message("\n\nERROR: Could not open interpolation output file.\n");
        return;
    }
    for(point=0; point<total_points_found; point++)
    {
        for(i=0; i<NSCALARS+ND_ND+1; i++)
        {
            fprintf(output, "%12.5f ", values[point][i]);
        }
        fprintf(output,"\n");
    }
    fprintf(output,"\n");
    fclose(output);

}


int is_point_in_cell(cell_t c, Thread *t, real *x)
{
    int i;
    face_t f;
    Thread *tf;
    real A[ND_ND], n[ND_ND], v[ND_ND], z[ND_ND];
    real face_centroid[ND_ND], cell_centroid[ND_ND];
    real Amag;

    /* Center of cell */
    C_CENTROID(cell_centroid, c, t);

    /* Loop over all faces of cell */
    c_face_loop(c, t, i)
    {
        /* Face i of cell */
        f = C_FACE(c, t, i);
        tf = C_FACE_THREAD(c, t, i);

        /* Face normal */
        F_AREA(A, f, tf);

        /* Normalize A to reduce truncation error */
        Amag = NV_MAG(A);
        NV_VS(n, =, A, /, Amag);

        /* Centroid on face i */
        F_CENTROID(face_centroid, f, tf);

        /* Vector from face centroid to point x */
        NV_VV(v, =, x, -, face_centroid);

        /* Vector from cell centroid to face centroid */
        NV_VV(z, =, face_centroid, -, cell_centroid);

        /* Perform test to make sure face normal points outwards */
        if (NV_DOT(n, z) < 0.0)
            NV_S(n, *=, -1.0);

        /* If n.v > 0, then point is beyond "plane" that defines 
           the face and it annot be inside the cell */
        if (NV_DOT(n, v) > 0.0)
            return 0;
    }
    
    /* Otherwise it must be in the cell */
    return 1;
}

\end{minted}

%%% Local Variables:
%%% mode: latex
%%% TeX-master: "interpolationUDF"
%%% End:
